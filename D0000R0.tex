% !TeX program = pdflatex
\documentclass{wg21}

\usepackage{xcolor}
\usepackage{soul}
\usepackage{ulem}
\usepackage{fullpage}
\usepackage{parskip}
\usepackage{listings}
\usepackage[cachedir=.build,outputdir=.build]{minted}
\usepackage{enumitem}
\usepackage{csquotes}

\lstdefinestyle{base}{
  language=c++,
  breaklines=false,
  basicstyle=\ttfamily\color{black},
  moredelim=**[is][\color{green!50!black}]{@}{@},
  escapeinside={(*@}{@*)}
}

\newcommand{\cc}[1]{\mintinline{c++}{#1}}
\newminted[cpp]{c++}{}


\title{fs::current\_thread\_path()}
\docnumber{DXXXXR0}
\audience{Library Evolution Group}
\author{Egor Pugin}{egor.pugin@gmail.com}

\begin{document}
\maketitle

\section{Revision history}
\begin{itemize}
  \item R0 -- Initial draft
\end{itemize}


\section{Abstract}
This paper proposes an addition of a function <<std::filesystem::current_thread_path()>> to filesystem library.

\begin{cpp}
namespace fs = std::filesystem;

fs::current_path();        // existing, returns process-wide working directory
fs::current_thread_path(); // proposed, returns thread-local working directory
\end{cpp}


\section{Motivation}
When application performs extensively on filesystem in multithreaded environment,\\ fs::current_directory() often is not enough.
Programmers have to implement some thread's current working directory via: 1) global thread local variable, 2) a function with thread local variable or 3) local/member variable that is passed down to users.

Implementation of this facility in many cases is more or less common, so it can be considered as an addition to Standard Library.



\section{Possible implementation}

% 30.11

fs::current_thread_path() gets it initial value from fs::current_path() or user-provided value on the first call.
Later, it can be retrieved or changed in the same way.

\subsection{Declaration}

\begin{cpp}
namespace std::filesystem
{

    path current_thread_path();
    path current_thread_path(std::error_code& ec);
    void current_thread_path(const path& p);
    void current_thread_path(const path& p, std::error_code& ec) noexcept;

}
\end{cpp}

\subsection{Definition}

\begin{cpp}
namespace std::filesystem
{

    path __current_thread_path(const path &p = path(), std::error_code* ec = nullptr)
    {
        thread_local auto thread_working_dir = ec ? current_path(*ec) : current_path();
        if (p.empty())
            return thread_working_dir;
        return thread_working_dir = fs::absolute(p);
    }
    
    path current_thread_path()
    {
        return __current_thread_path();
    }
    
    path current_thread_path(std::error_code& ec)
    {
        return __current_thread_path(path(), &ec);
    }
    
    void current_thread_path(const path& p)
    {
        __current_thread_path(p);
    }
    
    void current_thread_path(const path& p, std::error_code& ec) noexcept
    {
        __current_thread_path(p, &ec);
    }

}
\end{cpp}


\subsection{Other variants}

\begin{cpp}
std::this_thread::path()?
\end{cpp}


\section{Proposed wording}

TBD




\section{References}
\renewcommand{\section}[2]{}%
\begin{thebibliography}{9}
  \bibitem[N4727]{N4727}
    Richard Smith,
    \emph{Working Draft, Standard for Programming Language C++}\newline
    \url{http://www.open-std.org/jtc1/sc22/wg21/docs/papers/2018/n4727.pdf}

\end{thebibliography}

\end{document}